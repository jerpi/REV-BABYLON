\documentclass[10pt,a4paper]{article}
\usepackage[utf8]{inputenc}
\usepackage{amsmath}
\usepackage{amsfonts}
\usepackage{amssymb}
\usepackage{graphicx}

\title{ReV - projet}
\date{Automne 2017}

\begin{document}
\maketitle
	
	\section{Objectif}
	On cherche à réaliser un environnement 3d capable de réagir aux actions d'un utilisateur. L'application retenue est celle d'un musée dans lequel les oeuvres exposées vont sinon contrôler le visiteur, du moins le guider, en réagissant aux actions de celui-ci.

    \section{Description}
    Le musée a une emprise au sol de 30*30 $m^2$. Il est construit sur deux niveaux. Au premier niveau se trouvent : 
    \begin{itemize}
    	\item un hall : 30*15 $m^2$, placé au sud
    	\item trois salles : 10*15 $m^2$, placées au nord
    \end{itemize}
    %%
    Le second niveau est constitué par une mezzanine, au-dessus des trois salles.
    \\
    \\
    Le premier et le second niveau sont reliés par un escalier et un ascenseur.
    
    \section{Modélisation}
    
    \subsection{Structure}
    Proposez une modélisation du musée. Celle-ci sera réalisée de façon procédurale, sans utiliser de modeleurs ou de fichiers d'objet 3d. Les éléments 
    à modéliser contiennent les sols et les cloisons. 
    
    \subsubsection{Géométrie}
    Dans un premier temps seule la géométrie des objets est modélisée, c'est à dire pour chaque primitive géométrique : 
    
    \begin{itemize}
    	\item sa nature (sphère, boite, cylindre, ...)
    	\item les paramètres qui décrivent ses dimensions (rayon, taille, ...)
    	\item les paramètres de placement dans l'espace (translation, rotation)
    \end{itemize}
    
    \subsubsection{Aspect}
    Proposez un habillage des objets (matériaux et textures), aussi bien pour les sols que pour les cloisons.
    

    \subsection{Eléments muséographiques}
    
   
    
    \subsubsection{Tableaux}
    On place 8 tableaux par salle. Ces tableaux sont accrochés aux murs.
    \subsubsection{Sculptures}
    Une sculpture est un ensemble de primitives géométriques placées sur un podium (une boite). Certaines de ces sculptures doivent pouvoir être articulées.
    
    \subsubsection{Signalétique}
    Placez dans l'espace des pancartes qui 
    \begin{itemize}
    	\item donnent le thème de la salle
    	\item proposent des directions à suivre
    \end{itemize}
    Ainsi que des bancs placés devant des oeuvres intéressantes pour inciter les utilisateurs à s'y intéresser.
	
	\section{Eclairage}
	Il s'agit ici de proposer un éclairage du monde virtuel tout en respectant les contraintes de certaines versions de BABYLON : pas plus de simultannément  4 sources lumineuses 
	
	\subsection{Eclairage général}
	Proposez un éclairage permettant d'éclairer la totalité du musée (directionnel et ambient).
	
	\subsection{Eclairage local}
	Faîtes en sorte que lorsque l'utilisateur entre dans une salle, deux tableaux soient sur-éclairés (par rapport aux autres) afin de les mettre en évidence.
	
	\section{Physique}
	
	\subsection{Gravité}
	Ajoutez la gravité à la scène.
	
	\subsection{Collisions}
	Faîtes en sorte que l'utilisateur ne puisse pas traverser les objets, en particulier les cloisons.
	
	
	\section{Animation}
	\subsection{Visites virtuelles}
	Dans ces visites virtuelle l'utilisateur est passif : sa position et son orientation sont déterminés à tout moment par programme. Proposez plusieurs parcours possibles. La visite courante sera choisie de façon aléatoire.
	
	\subsection{Sculptures animées}
	Proposez une animation des sculptures articulées (elles sont animées mais restent sur place).
\end{document}